\documentclass[11pt,compress,t,notes=noshow, xcolor=table]{beamer}
\usepackage[]{graphicx}\usepackage[]{color}
% maxwidth is the original width if it is less than linewidth
% otherwise use linewidth (to make sure the graphics do not exceed the margin)
\makeatletter
\def\maxwidth{ %
  \ifdim\Gin@nat@width>\linewidth
    \linewidth
  \else
    \Gin@nat@width
  \fi
}
\makeatother

\definecolor{fgcolor}{rgb}{0.345, 0.345, 0.345}
\newcommand{\hlnum}[1]{\textcolor[rgb]{0.686,0.059,0.569}{#1}}%
\newcommand{\hlstr}[1]{\textcolor[rgb]{0.192,0.494,0.8}{#1}}%
\newcommand{\hlcom}[1]{\textcolor[rgb]{0.678,0.584,0.686}{\textit{#1}}}%
\newcommand{\hlopt}[1]{\textcolor[rgb]{0,0,0}{#1}}%
\newcommand{\hlstd}[1]{\textcolor[rgb]{0.345,0.345,0.345}{#1}}%
\newcommand{\hlkwa}[1]{\textcolor[rgb]{0.161,0.373,0.58}{\textbf{#1}}}%
\newcommand{\hlkwb}[1]{\textcolor[rgb]{0.69,0.353,0.396}{#1}}%
\newcommand{\hlkwc}[1]{\textcolor[rgb]{0.333,0.667,0.333}{#1}}%
\newcommand{\hlkwd}[1]{\textcolor[rgb]{0.737,0.353,0.396}{\textbf{#1}}}%
\let\hlipl\hlkwb

\usepackage{framed}
\makeatletter
\newenvironment{kframe}{%
 \def\at@end@of@kframe{}%
 \ifinner\ifhmode%
  \def\at@end@of@kframe{\end{minipage}}%
  \begin{minipage}{\columnwidth}%
 \fi\fi%
 \def\FrameCommand##1{\hskip\@totalleftmargin \hskip-\fboxsep
 \colorbox{shadecolor}{##1}\hskip-\fboxsep
     % There is no \\@totalrightmargin, so:
     \hskip-\linewidth \hskip-\@totalleftmargin \hskip\columnwidth}%
 \MakeFramed {\advance\hsize-\width
   \@totalleftmargin\z@ \linewidth\hsize
   \@setminipage}}%
 {\par\unskip\endMakeFramed%
 \at@end@of@kframe}
\makeatother

\definecolor{shadecolor}{rgb}{.97, .97, .97}
\definecolor{messagecolor}{rgb}{0, 0, 0}
\definecolor{warningcolor}{rgb}{1, 0, 1}
\definecolor{errorcolor}{rgb}{1, 0, 0}
\newenvironment{knitrout}{}{} % an empty environment to be redefined in TeX

\usepackage{alltt}
\newcommand{\SweaveOpts}[1]{}  % do not interfere with LaTeX
\newcommand{\SweaveInput}[1]{} % because they are not real TeX commands
\newcommand{\Sexpr}[1]{}       % will only be parsed by R
\newcommand{\xmark}{\ding{55}}%


\usepackage[english]{babel}
\usepackage[utf8]{inputenc}

\usepackage{dsfont}
\usepackage{verbatim}
\usepackage{amsmath}
\usepackage{amsfonts}
\usepackage{amssymb}
\usepackage{bm}
\usepackage{csquotes}
\usepackage{multirow}
\usepackage{longtable}
\usepackage{booktabs}
\usepackage{enumerate}
\usepackage[absolute,overlay]{textpos}
\usepackage{psfrag}
\usepackage{algorithm}
\usepackage{algpseudocode}
\usepackage{eqnarray}
\usepackage{arydshln}
\usepackage{tabularx}
\usepackage{placeins}
\usepackage{tikz}
\usepackage{setspace}
\usepackage{colortbl}
\usepackage{mathtools}
\usepackage{wrapfig}
\usepackage{bm}
\usepackage{amsmath}
\usepackage{pifont}

\usetikzlibrary{shapes,arrows,automata,positioning,calc,chains,trees, shadows}
\tikzset{
  %Define standard arrow tip
  >=stealth',
  %Define style for boxes
  punkt/.style={
    rectangle,
    rounded corners,
    draw=black, very thick,
    text width=6.5em,
    minimum height=2em,
    text centered},
  % Define arrow style
  pil/.style={
    ->,
    thick,
    shorten <=2pt,
    shorten >=2pt,}
}

\usepackage{subfig}

% Defines macros and environments
\usepackage{../../style/lmu-lecture}


\let\code=\texttt
\let\proglang=\textsf

\setkeys{Gin}{width=0.9\textwidth}

\setbeamertemplate{frametitle}{\expandafter\uppercase\expandafter\insertframetitle}

\usepackage{bbm}
% basic latex stuff
\newcommand{\pkg}[1]{{\fontseries{b}\selectfont #1}} %fontstyle for R packages
\newcommand{\lz}{\vspace{0.5cm}} %vertical space
\newcommand{\dlz}{\vspace{1cm}} %double vertical space
\newcommand{\oneliner}[1] % Oneliner for important statements
{\begin{block}{}\begin{center}\begin{Large}#1\end{Large}\end{center}\end{block}}


%new environments
\newenvironment{vbframe}  %frame with breaks and verbatim
{
 \begin{frame}[containsverbatim,allowframebreaks]
}
{
\end{frame}
}

\newenvironment{vframe}  %frame with verbatim without breaks (to avoid numbering one slided frames)
{
 \begin{frame}[containsverbatim]
}
{
\end{frame}
}

\newenvironment{blocki}[1]   % itemize block
{
 \begin{block}{#1}\begin{itemize}
}
{
\end{itemize}\end{block}
}

\newenvironment{fragileframe}[2]{  %fragile frame with framebreaks
\begin{frame}[allowframebreaks, fragile, environment = fragileframe]
\frametitle{#1}
#2}
{\end{frame}}


\newcommand{\myframe}[2]{  %short for frame with framebreaks
\begin{frame}[allowframebreaks]
\frametitle{#1}
#2
\end{frame}}

\newcommand{\remark}[1]{
  \textbf{Remark:} #1
}

\newcommand{\citebutton}[2]{%
\NoCaseChange{\resizebox{!}{9pt}{\protect\beamergotobutton{\href{#2}{#1}}}}%
}



\newenvironment{deleteframe}
{
\begingroup
\usebackgroundtemplate{\includegraphics[width=\paperwidth,height=\paperheight]{../style/color/red.png}}
 \begin{frame}
}
{
\end{frame}
\endgroup
}
\newenvironment{simplifyframe}
{
\begingroup
\usebackgroundtemplate{\includegraphics[width=\paperwidth,height=\paperheight]{../style/color/yellow.png}}
 \begin{frame}
}
{
\end{frame}
\endgroup
}\newenvironment{draftframe}
{
\begingroup
\usebackgroundtemplate{\includegraphics[width=\paperwidth,height=\paperheight]{../style/color/green.jpg}}
 \begin{frame}
}
{
\end{frame}
\endgroup
}
% https://tex.stackexchange.com/a/261480: textcolor that works in mathmode
\makeatletter
\renewcommand*{\@textcolor}[3]{%
  \protect\leavevmode
  \begingroup
    \color#1{#2}#3%
  \endgroup
}
\makeatother





% math spaces
\ifdefined\N                                                                
\renewcommand{\N}{\mathds{N}} % N, naturals
\else \newcommand{\N}{\mathds{N}} \fi 
\newcommand{\Z}{\mathds{Z}} % Z, integers
\newcommand{\Q}{\mathds{Q}} % Q, rationals
\newcommand{\R}{\mathds{R}} % R, reals
\ifdefined\C 
  \renewcommand{\C}{\mathds{C}} % C, complex
\else \newcommand{\C}{\mathds{C}} \fi
\newcommand{\continuous}{\mathcal{C}} % C, space of continuous functions
\newcommand{\M}{\mathcal{M}} % machine numbers
\newcommand{\epsm}{\epsilon_m} % maximum error

% counting / finite sets
\newcommand{\setzo}{\{0, 1\}} % set 0, 1
\newcommand{\setmp}{\{-1, +1\}} % set -1, 1
\newcommand{\unitint}{[0, 1]} % unit interval

% basic math stuff
\newcommand{\xt}{\tilde x} % x tilde
\newcommand{\argmax}{\operatorname{arg\,max}} % argmax
\newcommand{\argmin}{\operatorname{arg\,min}} % argmin
\newcommand{\argminlim}{\mathop{\mathrm{arg\,min}}\limits} % argmax with limits
\newcommand{\argmaxlim}{\mathop{\mathrm{arg\,max}}\limits} % argmin with limits  
\newcommand{\sign}{\operatorname{sign}} % sign, signum
\newcommand{\I}{\mathbb{I}} % I, indicator
\newcommand{\order}{\mathcal{O}} % O, order
\newcommand{\pd}[2]{\frac{\partial{#1}}{\partial #2}} % partial derivative
\newcommand{\floorlr}[1]{\left\lfloor #1 \right\rfloor} % floor
\newcommand{\ceillr}[1]{\left\lceil #1 \right\rceil} % ceiling

% sums and products
\newcommand{\sumin}{\sum\limits_{i=1}^n} % summation from i=1 to n
\newcommand{\sumim}{\sum\limits_{i=1}^m} % summation from i=1 to m
\newcommand{\sumjn}{\sum\limits_{j=1}^n} % summation from j=1 to p
\newcommand{\sumjp}{\sum\limits_{j=1}^p} % summation from j=1 to p
\newcommand{\sumik}{\sum\limits_{i=1}^k} % summation from i=1 to k
\newcommand{\sumkg}{\sum\limits_{k=1}^g} % summation from k=1 to g
\newcommand{\sumjg}{\sum\limits_{j=1}^g} % summation from j=1 to g
\newcommand{\meanin}{\frac{1}{n} \sum\limits_{i=1}^n} % mean from i=1 to n
\newcommand{\meanim}{\frac{1}{m} \sum\limits_{i=1}^m} % mean from i=1 to n
\newcommand{\meankg}{\frac{1}{g} \sum\limits_{k=1}^g} % mean from k=1 to g
\newcommand{\prodin}{\prod\limits_{i=1}^n} % product from i=1 to n
\newcommand{\prodkg}{\prod\limits_{k=1}^g} % product from k=1 to g
\newcommand{\prodjp}{\prod\limits_{j=1}^p} % product from j=1 to p

% linear algebra
\newcommand{\one}{\boldsymbol{1}} % 1, unitvector
\newcommand{\zero}{\mathbf{0}} % 0-vector
\newcommand{\id}{\boldsymbol{I}} % I, identity
\newcommand{\diag}{\operatorname{diag}} % diag, diagonal
\newcommand{\trace}{\operatorname{tr}} % tr, trace
\newcommand{\spn}{\operatorname{span}} % span
\newcommand{\scp}[2]{\left\langle #1, #2 \right\rangle} % <.,.>, scalarproduct
\newcommand{\mat}[1]{\begin{pmatrix} #1 \end{pmatrix}} % short pmatrix command
\newcommand{\Amat}{\mathbf{A}} % matrix A
\newcommand{\Deltab}{\mathbf{\Delta}} % error term for vectors

% basic probability + stats
\renewcommand{\P}{\mathds{P}} % P, probability
\newcommand{\E}{\mathds{E}} % E, expectation
\newcommand{\var}{\mathsf{Var}} % Var, variance
\newcommand{\cov}{\mathsf{Cov}} % Cov, covariance
\newcommand{\corr}{\mathsf{Corr}} % Corr, correlation
\newcommand{\normal}{\mathcal{N}} % N of the normal distribution
\newcommand{\iid}{\overset{i.i.d}{\sim}} % dist with i.i.d superscript
\newcommand{\distas}[1]{\overset{#1}{\sim}} % ... is distributed as ...

% machine learning
\newcommand{\Xspace}{\mathcal{X}} % X, input space
\newcommand{\Yspace}{\mathcal{Y}} % Y, output space
\newcommand{\nset}{\{1, \ldots, n\}} % set from 1 to n
\newcommand{\pset}{\{1, \ldots, p\}} % set from 1 to p
\newcommand{\gset}{\{1, \ldots, g\}} % set from 1 to g
\newcommand{\Pxy}{\mathbb{P}_{xy}} % P_xy
\newcommand{\Exy}{\mathbb{E}_{xy}} % E_xy: Expectation over random variables xy
\newcommand{\xv}{\mathbf{x}} % vector x (bold)
\newcommand{\xtil}{\tilde{\mathbf{x}}} % vector x-tilde (bold)
\newcommand{\yv}{\mathbf{y}} % vector y (bold)
\newcommand{\xy}{(\xv, y)} % observation (x, y)
\newcommand{\xvec}{\left(x_1, \ldots, x_p\right)^\top} % (x1, ..., xp) 
\newcommand{\Xmat}{\mathbf{X}} % Design matrix
\newcommand{\allDatasets}{\mathds{D}} % The set of all datasets
\newcommand{\allDatasetsn}{\mathds{D}_n}  % The set of all datasets of size n 
\newcommand{\D}{\mathcal{D}} % D, data
\newcommand{\Dn}{\D_n} % D_n, data of size n
\newcommand{\Dtrain}{\mathcal{D}_{\text{train}}} % D_train, training set
\newcommand{\Dtest}{\mathcal{D}_{\text{test}}} % D_test, test set
\newcommand{\xyi}[1][i]{\left(\xv^{(#1)}, y^{(#1)}\right)} % (x^i, y^i), i-th observation
\newcommand{\Dset}{\left( \xyi[1], \ldots, \xyi[n]\right)} % {(x1,y1)), ..., (xn,yn)}, data
\newcommand{\defAllDatasetsn}{(\Xspace \times \Yspace)^n} % Def. of the set of all datasets of size n 
\newcommand{\defAllDatasets}{\bigcup_{n \in \N}(\Xspace \times \Yspace)^n} % Def. of the set of all datasets 
\newcommand{\xdat}{\left\{ \xv^{(1)}, \ldots, \xv^{(n)}\right\}} % {x1, ..., xn}, input data
\newcommand{\yvec}{\left(y^{(1)}, \hdots, y^{(n)}\right)^\top} % (y1, ..., yn), vector of outcomes
\renewcommand{\xi}[1][i]{\xv^{(#1)}} % x^i, i-th observed value of x
\newcommand{\yi}[1][i]{y^{(#1)}} % y^i, i-th observed value of y 
\newcommand{\xivec}{\left(x^{(i)}_1, \ldots, x^{(i)}_p\right)^\top} % (x1^i, ..., xp^i), i-th observation vector
\newcommand{\xj}{\xv_j} % x_j, j-th feature
\newcommand{\xjvec}{\left(x^{(1)}_j, \ldots, x^{(n)}_j\right)^\top} % (x^1_j, ..., x^n_j), j-th feature vector
\newcommand{\phiv}{\mathbf{\phi}} % Basis transformation function phi
\newcommand{\phixi}{\mathbf{\phi}^{(i)}} % Basis transformation of xi: phi^i := phi(xi)

%%%%%% ml - models general
\newcommand{\lamv}{\bm{\lambda}} % lambda vector, hyperconfiguration vector
\newcommand{\Lam}{\bm{\Lambda}}	 % Lambda, space of all hpos
% Inducer / Inducing algorithm
\newcommand{\preimageInducer}{\left(\defAllDatasets\right)\times\Lam} % Set of all datasets times the hyperparameter space
\newcommand{\preimageInducerShort}{\allDatasets\times\Lam} % Set of all datasets times the hyperparameter space
% Inducer / Inducing algorithm
\newcommand{\ind}{\mathcal{I}} % Inducer, inducing algorithm, learning algorithm 

% continuous prediction function f
\newcommand{\ftrue}{f_{\text{true}}}  % True underlying function (if a statistical model is assumed)
\newcommand{\ftruex}{\ftrue(\xv)} % True underlying function (if a statistical model is assumed)
\newcommand{\fx}{f(\xv)} % f(x), continuous prediction function
\newcommand{\fdomains}{f: \Xspace \rightarrow \R^g} % f with domain and co-domain
\newcommand{\Hspace}{\mathcal{H}} % hypothesis space where f is from
\newcommand{\fbayes}{f^{\ast}} % Bayes-optimal model
\newcommand{\fxbayes}{f^{\ast}(\xv)} % Bayes-optimal model
\newcommand{\fkx}[1][k]{f_{#1}(\xv)} % f_j(x), discriminant component function
\newcommand{\fh}{\hat{f}} % f hat, estimated prediction function
\newcommand{\fxh}{\fh(\xv)} % fhat(x)
\newcommand{\fxt}{f(\xv ~|~ \thetab)} % f(x | theta)
\newcommand{\fxi}{f\left(\xv^{(i)}\right)} % f(x^(i))
\newcommand{\fxih}{\hat{f}\left(\xv^{(i)}\right)} % f(x^(i))
\newcommand{\fxit}{f\left(\xv^{(i)} ~|~ \thetab\right)} % f(x^(i) | theta)
\newcommand{\fhD}{\fh_{\D}} % fhat_D, estimate of f based on D
\newcommand{\fhDtrain}{\fh_{\Dtrain}} % fhat_Dtrain, estimate of f based on D
\newcommand{\fhDnlam}{\fh_{\Dn, \lamv}} %model learned on Dn with hp lambda
\newcommand{\fhDlam}{\fh_{\D, \lamv}} %model learned on D with hp lambda
\newcommand{\fhDnlams}{\fh_{\Dn, \lamv^\ast}} %model learned on Dn with optimal hp lambda 
\newcommand{\fhDlams}{\fh_{\D, \lamv^\ast}} %model learned on D with optimal hp lambda 

% discrete prediction function h
\newcommand{\hx}{h(\xv)} % h(x), discrete prediction function
\newcommand{\hh}{\hat{h}} % h hat
\newcommand{\hxh}{\hat{h}(\xv)} % hhat(x)
\newcommand{\hxt}{h(\xv | \thetab)} % h(x | theta)
\newcommand{\hxi}{h\left(\xi\right)} % h(x^(i))
\newcommand{\hxit}{h\left(\xi ~|~ \thetab\right)} % h(x^(i) | theta)
\newcommand{\hbayes}{h^{\ast}} % Bayes-optimal classification model
\newcommand{\hxbayes}{h^{\ast}(\xv)} % Bayes-optimal classification model

% yhat
\newcommand{\yh}{\hat{y}} % yhat for prediction of target
\newcommand{\yih}{\hat{y}^{(i)}} % yhat^(i) for prediction of ith targiet
\newcommand{\resi}{\yi- \yih}

% theta
\newcommand{\thetah}{\hat{\theta}} % theta hat
\newcommand{\thetab}{\bm{\theta}} % theta vector
\newcommand{\thetabh}{\bm{\hat\theta}} % theta vector hat
\newcommand{\thetat}[1][t]{\thetab^{[#1]}} % theta^[t] in optimization
\newcommand{\thetatn}[1][t]{\thetab^{[#1 +1]}} % theta^[t+1] in optimization
\newcommand{\thetahDnlam}{\thetabh_{\Dn, \lamv}} %theta learned on Dn with hp lambda
\newcommand{\thetahDlam}{\thetabh_{\D, \lamv}} %theta learned on D with hp lambda
\newcommand{\mint}{\min_{\thetab \in \Theta}} % min problem theta
\newcommand{\argmint}{\argmin_{\thetab \in \Theta}} % argmin theta

% densities + probabilities
% pdf of x 
\newcommand{\pdf}{p} % p
\newcommand{\pdfx}{p(\xv)} % p(x)
\newcommand{\pixt}{\pi(\xv~|~ \thetab)} % pi(x|theta), pdf of x given theta
\newcommand{\pixit}{\pi\left(\xi ~|~ \thetab\right)} % pi(x^i|theta), pdf of x given theta
\newcommand{\pixii}{\pi\left(\xi\right)} % pi(x^i), pdf of i-th x 

% pdf of (x, y)
\newcommand{\pdfxy}{p(\xv,y)} % p(x, y)
\newcommand{\pdfxyt}{p(\xv, y ~|~ \thetab)} % p(x, y | theta)
\newcommand{\pdfxyit}{p\left(\xi, \yi ~|~ \thetab\right)} % p(x^(i), y^(i) | theta)

% pdf of x given y
\newcommand{\pdfxyk}[1][k]{p(\xv | y= #1)} % p(x | y = k)
\newcommand{\lpdfxyk}[1][k]{\log p(\xv | y= #1)} % log p(x | y = k)
\newcommand{\pdfxiyk}[1][k]{p\left(\xi | y= #1 \right)} % p(x^i | y = k)

% prior probabilities
\newcommand{\pik}[1][k]{\pi_{#1}} % pi_k, prior
\newcommand{\lpik}[1][k]{\log \pi_{#1}} % log pi_k, log of the prior
\newcommand{\pit}{\pi(\thetab)} % Prior probability of parameter theta

% posterior probabilities
\newcommand{\post}{\P(y = 1 ~|~ \xv)} % P(y = 1 | x), post. prob for y=1
\newcommand{\postk}[1][k]{\P(y = #1 ~|~ \xv)} % P(y = k | y), post. prob for y=k
\newcommand{\pidomains}{\pi: \Xspace \rightarrow \unitint} % pi with domain and co-domain
\newcommand{\pibayes}{\pi^{\ast}} % Bayes-optimal classification model
\newcommand{\pixbayes}{\pi^{\ast}(\xv)} % Bayes-optimal classification model
\newcommand{\pix}{\pi(\xv)} % pi(x), P(y = 1 | x)
\newcommand{\pikx}[1][k]{\pi_{#1}(\xv)} % pi_k(x), P(y = k | x)
\newcommand{\pikxt}[1][k]{\pi_{#1}(\xv ~|~ \thetab)} % pi_k(x | theta), P(y = k | x, theta)
\newcommand{\pixh}{\hat \pi(\xv)} % pi(x) hat, P(y = 1 | x) hat
\newcommand{\pikxh}[1][k]{\hat \pi_{#1}(\xv)} % pi_k(x) hat, P(y = k | x) hat
\newcommand{\pixih}{\hat \pi(\xi)} % pi(x^(i)) with hat
\newcommand{\pikxih}[1][k]{\hat \pi_{#1}(\xi)} % pi_k(x^(i)) with hat
\newcommand{\pdfygxt}{p(y ~|~\xv, \thetab)} % p(y | x, theta)
\newcommand{\pdfyigxit}{p\left(\yi ~|~\xi, \thetab\right)} % p(y^i |x^i, theta)
\newcommand{\lpdfygxt}{\log \pdfygxt } % log p(y | x, theta)
\newcommand{\lpdfyigxit}{\log \pdfyigxit} % log p(y^i |x^i, theta)

% probababilistic
\newcommand{\bayesrulek}[1][k]{\frac{\P(\xv | y= #1) \P(y= #1)}{\P(\xv)}} % Bayes rule
\newcommand{\muk}{\bm{\mu_k}} % mean vector of class-k Gaussian (discr analysis) 

% residual and margin
\newcommand{\eps}{\epsilon} % residual, stochastic
\newcommand{\epsi}{\epsilon^{(i)}} % epsilon^i, residual, stochastic
\newcommand{\epsh}{\hat{\epsilon}} % residual, estimated
\newcommand{\yf}{y \fx} % y f(x), margin
\newcommand{\yfi}{\yi \fxi} % y^i f(x^i), margin
\newcommand{\Sigmah}{\hat \Sigma} % estimated covariance matrix
\newcommand{\Sigmahj}{\hat \Sigma_j} % estimated covariance matrix for the j-th class

% ml - loss, risk, likelihood
\newcommand{\Lyf}{L\left(y, f\right)} % L(y, f), loss function
\newcommand{\Lxy}{L\left(y, \fx\right)} % L(y, f(x)), loss function
\newcommand{\Lxyi}{L\left(\yi, \fxi\right)} % loss of observation
\newcommand{\Lxyt}{L\left(y, \fxt\right)} % loss with f parameterized
\newcommand{\Lxyit}{L\left(\yi, \fxit\right)} % loss of observation with f parameterized
\newcommand{\Lxym}{L\left(\yi, f\left(\bm{\tilde{x}}^{(i)} ~|~ \thetab\right)\right)} % loss of observation with f parameterized
\newcommand{\Lpixy}{L\left(y, \pix\right)} % loss in classification
\newcommand{\Lpixyi}{L\left(\yi, \pixii\right)} % loss of observation in classification
\newcommand{\Lpixyt}{L\left(y, \pixt\right)} % loss with pi parameterized
\newcommand{\Lpixyit}{L\left(\yi, \pixit\right)} % loss of observation with pi parameterized
\newcommand{\Lhxy}{L\left(y, \hx\right)} % L(y, h(x)), loss function on discrete classes
\newcommand{\Lr}{L\left(r\right)} % L(r), loss defined on residual (reg) / margin (classif)
\newcommand{\lone}{|y - \fx|} % L1 loss
\newcommand{\ltwo}{\left(y - \fx\right)^2} % L2 loss
\newcommand{\lbernoullimp}{\ln(1 + \exp(-y \cdot \fx))} % Bernoulli loss for -1, +1 encoding
\newcommand{\lbernoullizo}{- y \cdot \fx + \log(1 + \exp(\fx))} % Bernoulli loss for 0, 1 encoding
\newcommand{\lcrossent}{- y \log \left(\pix\right) - (1 - y) \log \left(1 - \pix\right)} % cross-entropy loss
\newcommand{\lbrier}{\left(\pix - y \right)^2} % Brier score
\newcommand{\risk}{\mathcal{R}} % R, risk
\newcommand{\riskbayes}{\mathcal{R}^\ast}
\newcommand{\riskf}{\risk(f)} % R(f), risk
\newcommand{\riskdef}{\E_{y|\xv}\left(\Lxy \right)} % risk def (expected loss)
\newcommand{\riskt}{\mathcal{R}(\thetab)} % R(theta), risk
\newcommand{\riske}{\mathcal{R}_{\text{emp}}} % R_emp, empirical risk w/o factor 1 / n
\newcommand{\riskeb}{\bar{\mathcal{R}}_{\text{emp}}} % R_emp, empirical risk w/ factor 1 / n
\newcommand{\riskef}{\riske(f)} % R_emp(f)
\newcommand{\risket}{\mathcal{R}_{\text{emp}}(\thetab)} % R_emp(theta)
\newcommand{\riskr}{\mathcal{R}_{\text{reg}}} % R_reg, regularized risk
\newcommand{\riskrt}{\mathcal{R}_{\text{reg}}(\thetab)} % R_reg(theta)
\newcommand{\riskrf}{\riskr(f)} % R_reg(f)
\newcommand{\riskrth}{\hat{\mathcal{R}}_{\text{reg}}(\thetab)} % hat R_reg(theta)
\newcommand{\risketh}{\hat{\mathcal{R}}_{\text{emp}}(\thetab)} % hat R_emp(theta)
\newcommand{\LL}{\mathcal{L}} % L, likelihood
\newcommand{\LLt}{\mathcal{L}(\thetab)} % L(theta), likelihood
\newcommand{\LLtx}{\mathcal{L}(\thetab | \xv)} % L(theta|x), likelihood
\newcommand{\logl}{\ell} % l, log-likelihood
\newcommand{\loglt}{\logl(\thetab)} % l(theta), log-likelihood
\newcommand{\logltx}{\logl(\thetab | \xv)} % l(theta|x), log-likelihood
\newcommand{\errtrain}{\text{err}_{\text{train}}} % training error
\newcommand{\errtest}{\text{err}_{\text{test}}} % test error
\newcommand{\errexp}{\overline{\text{err}_{\text{test}}}} % avg training error

% lm
\newcommand{\thx}{\thetab^\top \xv} % linear model
\newcommand{\olsest}{(\Xmat^\top \Xmat)^{-1} \Xmat^\top \yv} % OLS estimator in LM 


%\usepackage{algorithm}
%\usepackage{algorithmic}

\newcommand{\sens}{\mathbf{A}} % vector x (bold)
\newcommand{\ba}{\mathbf{a}}
\newcommand{\batilde}{\tilde{\mathbf{a}}}
\newcommand{\Px}{\mathbb{P}_{x}} % P_x
\newcommand{\Pxj}{\mathbb{P}_{x_j}} % P_{x_j}
\newcommand{\indep}{\perp \!\!\! \perp} % independence symbol
% ml - ROC
\newcommand{\np}{n_{+}} % no. of positive instances
\newcommand{\nn}{n_{-}} % no. of negative instances
\newcommand{\rn}{\pi_{-}} % proportion negative instances
\newcommand{\rp}{\pi_{+}} % proportion negative instances
% true/false pos/neg:
\newcommand{\tp}{\# \text{TP}} % true pos
\newcommand{\fap}{\# \text{FP}} % false pos (fp taken for partial derivs)
\newcommand{\tn}{\# \text{TN}} % true neg
\newcommand{\fan}{\# \text{FN}} % false neg

\usepackage{multicol}

\newcommand{\titlefigure}{figure/cost_matrix}
\newcommand{\learninggoals}{
  \item Learn the modelling approach via the cost matrix in cost-sensitive learning settings
  \item Understand the connection between cost-sensitive and imbalanced learning
  \item Get to know MetaCost as a general approach to make classifiers cost-sensitve
}

\title{Advanced Machine Learning}
\date{}

\begin{document}

\lecturechapter{Imbalanced Learning via Cost-Sensitive Learning}
\lecture{Advanced Machine Learning}



\sloppy


\begin{vbframe}{Cost-Sensitive learning: In a Nutshell}
	%	
	\scriptsize{
		%
		%
		\begin{itemize}
%			
		\item Cost-sensitive learning is a learning paradigm, where different (mis-)classification costs are taken into consideration and the learner seeks to minimize the total costs in expectation.
%
		\item Thus, the major difference to the ``classical'' cost-insensitive learning setting is that cost-sensitive learning deals differently with misclassifications. Most of the algorithms of the latter kind assume that the data sets are balanced, and all errors have the same cost.
		
%
		\item This is motivated by a plethora of real-world applications, where different costs between misclassifications are present:
%		
		\begin{itemize}
			\scriptsize
%			
			\item Medicine --- Misdiagnosing a cancer patient as healthy vs.\ misdiagnosing a healthy patient as having cancer (and then check again).
			%					
			\item Tracking criminals ---  Classify an innocent person as a terrorist vs.\ overlooking a terrorist.
			%			
			\item (Extreme) Weather prediction ---  Incorrectly predicting that no hurricane occurs vs.\ predicting a strong wind as a hurricane.
%			
			\item Credit granting --- Lending to a risky client vs. not lending to a trustworthy client.
			%			
			\item $\ldots$
%    	
%			
		\end{itemize}
%	
		\item In all these examples the costs of a false negative is much higher than the costs of a false positive.
%
		\end{itemize}
		
	%
	}
\end{vbframe}


\begin{vbframe}{Cost matrix}
%	
\scriptsize{
%	
	\begin{itemize}
%		
		\item In cost-sensitive learning we are provided with a cost matrix $\mathbf{C}$ of the form
%		
	\end{itemize}
	%	
	\begin{center}
		\tiny
		\begin{tabular}{cc|>{\centering\arraybackslash}p{8em}>{\centering\arraybackslash}p{8em}>{\centering\arraybackslash}p{5em}>{\centering\arraybackslash}p{8em}}
			& & \multicolumn{4}{c}{\bfseries True Class $y$} \\
			&  & $1$ & $2$ & $\ldots$ & $g$  \\
			\hline
			\bfseries Classification     & $1$ & $C(1,1)$  &  $C(1,2)$  & $\ldots$ &  $C(1,g)$ \\
			& & (True 1's) & (False 1's for 2's) & $\ldots$ &  (False 1's for $g$'s)  \\
			& $2$ &  $C(2,1)$  &  $C(2,2)$  & $\ldots$ & $C(2,g)$  \\
			$\yh$ & & (False 2's for 1's) & (True 2's) & $\ldots$ &  (False 2's for $g$'s)  \\
			& $\vdots$ & $\vdots$ & $\vdots$ & $\ldots$ & $\vdots$ \\
			& $g$ & $C(g,1)$ & $C(g,2)$  & $\ldots$ &  $C(g,g)$\\
			& & (False $g$'s for 1's) & (False $g$'s for 2's) & $\ldots$ &  (True $g$'s)  \\
		\end{tabular}
	\end{center}
	%	
	\begin{itemize}
		%		
		\item 	Here, $C(i,j)$ is the cost of classifying $j$ as $i,$ which in the cost-insensitive learning case is simply $C(i,j) = \mathds{1}_{[ i \neq j ]},$ i.e., each misclassification has the same cost of 1.
		%		
		
		\item Sometimes the cost matrix is provided by the application at hand, e.g.\ in the credit granting example, or provided by a domain expert. In many cases, however, the cost matrix needs to be estimated. This is usually done by using a heuristic or by learning a proper cost matrix from the training data.
%		
		\item The cost matrix is essential for the learning process, as
%		
		\begin{enumerate}
%			
			\scriptsize
%			
		\item too low costs might not change the decision boundaries significantly leading to (still) costly predictions,
%		
		\item too high costs might harm the generalization capability of the classifier on costly classes.
%			
		\end{enumerate}

%		
%		
	\end{itemize}

}
\end{vbframe}


\begin{vbframe}{Cost matrix for Imbalanced Learning}
	%	
	\footnotesize{
		%	
		\begin{itemize}
			%		
%			
			\item For imbalanced data sets biases towards majority classes can be enlarged or even too strong biases towards the minority can be created if the cost matrix is set inappropriately.
%			 
			\item A common heuristic for imbalanced data sets is to use 
%			
			\begin{itemize}
%				
				\footnotesize \item the imbalance ratio between majority and minority classes for misclassifying a minority class $j$ as a majority class $i$, i.e., $C(i,j) = \frac{n_i}{n_j}$ for classes $i$ and $j$ such that $n_j  \ll n_i,$  
%				
				\item and costs of 1 for misclassifying a majority class $j$ as a minority class $i$, i.e., $C(i,j) = 1$ for classes $i$ and $j$ such that $n_i \ll n_j.$ 
%				
				\item Usually, the costs of a correct classification is set to 0, which is justified also from a theoretical point of view as we will see later.
%				
			\end{itemize}
%
		\begin{minipage}{0.45\textwidth}    
					\item In an imbalanced binary classification problem we obtain the following cost matrix using this heuristic:
		\end{minipage}
%		
		\begin{minipage}{0.35\textwidth}    
						\hfill		
				\begin{tabular}{cc|cc}
					& &\multicolumn{2}{c}{True class} \\
					& & $y=1$ & $y=-1$  \\
					\hline
					\multirow{2}{*}{\parbox{0.3cm}{Pred.  class}}& $\hat y$ = 1     & $0$                & $ 1 $\\
					& $\hat y$ = -1 & $ \frac{n_-}{n_+} $              &  $0$   \\
				\end{tabular}
		\end{minipage}
			%		
	\item Thus, this heuristic is consistent with the general real case that the cost of false negatives is much higher than the cost of false positives.	
%	
%	\item Although this is a simple heuristic which provides a cost matrix very quickly, it has a couple of drawbacks.
%	
	\end{itemize}
		
	}
\end{vbframe}


\begin{vbframe}{Minimum expected Cost Principle}
	%	
	\footnotesize{
		%	

		\begin{itemize}\footnotesize
			%		
			\item Suppose we are provided with a cost matrix $\mathbf{C}$ and also we have knowledge of the true posterior distribution $p(\cdot ~|~ \xv).$ The most natural way to classify a given feature $\xv$ is then by following the \emph{minimum expected cost principle}.
%			
			\item Minimum expected cost principle: Use the class for prediction with the smallest expected costs, where the expected costs of a class $i\in\{1,\ldots,g\}$ is
%			
			$$ 	\E_{J \sim p(\cdot ~|~ \xv)}( C(i,J) ) = \sum_{j=1}^g 	p(j ~|~ \xv) C(i,j).	$$
			%
			\item Thus, if we have a classifier $f$ which uses a probabilistic score function $\pi:\Xspace \to [0,1]^g$ with $\pi(\xv) = (\pi(\xv)_1,\ldots,\pi(\xv)_g)^\top$ and $\sum_{j=1}^g \pi(\xv)_j = 1$ for the classification, then one can easily modify $f$ to take the expected costs into account:
%			
			$$  \tilde f (\xv) = \argmin_{i=1,\ldots,g} \sum_{j=1}^g 	\pi(\xv)_j C(i,j). $$
%					
		\end{itemize}
	}
\end{vbframe}


\begin{vbframe}{Minimum expected Cost Principle: Binary Case}
	%	
	\footnotesize{
		%		
		\begin{itemize}\footnotesize
			%		
			\item In the binary classification setting (i.e., $\Yspace = \{-1,1\}$) the minimum expected costs principle translates to predict the positive class if 
%			
			\begin{align*}
%				
					&\E_{J \sim p(\cdot ~|~ \xv)}( C(1,J) )  \leq \E_{J \sim p(\cdot ~|~ \xv)}( C(-1,J) ) \\
%
				 	&\Leftrightarrow p(-1 ~|~ \xv ) C(1,-1)  + 	p(1 ~|~ \xv ) C(1,1) \\ &
				 	\qquad \leq  p(-1 ~|~ \xv ) C(-1,-1)  + 	p(1 ~|~ \xv ) C(-1,1)  \\
%				 	
					&\Leftrightarrow p(-1 ~|~ \xv ) \left( C(1,-1) - C(-1,-1) \right)  \leq  	p(1 ~|~ \xv ) \left( C(-1,1) -C(1,1)\right)  
%				
			\end{align*}
			%			
			\item Note that the decision for predicting the positive class does not change if we use instead of $\mathbf{C}$ the simpler cost matrix $\mathbf{C}_{simple}$, where 
			\begin{itemize}
				\footnotesize
				 \item $C_{simple}(-1,-1)=C_{simple}(1,1) = 0$ 
				 \item  $C_{simple}(1,-1) =  C(1,-1) - C(-1,-1) $ 
				 \item $C_{simple}(-1,1) = C(-1,1) -C(1,1).$
			\end{itemize} 
%		
		\item Thus, one can assume without loss of generality that the cost matrix $\mathbf{C}$
		\lz
		
%		\centerline{Cost matrix }
		\begin{tabular}{cc|cc}
			& &\multicolumn{2}{c}{True class} \\
			& & $y=1$ & $y=-1$  \\
			\hline
			\multirow{2}{*}{\parbox{0.3cm}{Pred.  class}}& $\hat y$ = 1     & $C(1,1)$                & $C(1,-1)$\\
			& $\hat y$ = -1 & $C(-1,1)$              &  $C(-1,-1)$   \\
		\end{tabular}
%		

		\lz
		is of a simpler form  $\mathbf{C}_{simple}$:
		\lz
		\lz
		
%		
%		\centerline{Simple Cost matrix $\mathbf{C}_{simple}$}
		\begin{tabular}{cc|cc}
			& &\multicolumn{2}{c}{True class} \\
			& & $y=1$ & $y=-1$  \\
			\hline
			\multirow{2}{*}{\parbox{0.3cm}{Pred.  class}}& $\hat y$ = 1     & 0                 & $C(1,-1) - C(-1,-1) $\\
			& $\hat y$ = -1 & $C(-1,1) -C(1,1)$              & 0\\
		\end{tabular}

		\lz
		\item An analogous result can be shown for the multiclass setting.
		
		
		\item With this simpler cost matrix (relabeling $\mathbf{C}$ by $\mathbf{C}_{simple}$), the decision to predict the positive class boils down to 
%		
		\begin{align*}
%			 	
			& p(-1 ~|~ \xv ) C(1,-1)   \leq  	p(1 ~|~ \xv ) C(-1,1)  \\
%			
			&\Leftrightarrow (1- p(1 ~|~ \xv ) ) C(1,-1)   \leq  	p(1 ~|~ \xv ) C(-1,1) \\
%			
			&\Leftrightarrow \underbrace{\frac{C(1,-1)}{C(1,-1) + C(-1,1) } }_{=:c^*}  \leq  	p(1 ~|~ \xv ) \\
			%				
		\end{align*}
%	
		\item This yields the optimal threshold value $c^*$ for probabilistic score classifiers, so that any probabilistic classifier $f$ using a probabilistic score $\pi:\Xspace \to [0,1]$ can be modified to
%		
		$$   \tilde f(\xv) = 2 \cdot \mathds{1}_{[ \pi(\xv) \geq c^*]} -1. $$
							
		\end{itemize}
	}
\end{vbframe}

%%	\begin{itemize}
	%		
	%		
	%		%\begin{center}
	%		\centerline{Confusion matrix}
	%		\begin{tabular}{cc|cc}
		%			& &\multicolumn{2}{c}{True class} \\
		%			& & $y=1$ & $y=-1$  \\
		%			\hline
		%			\multirow{2}{*}{\parbox{0.3cm}{Pred.  class}}& $\hat y$ = 1     & TP                 & FP\\
		%			& $\hat y$ = -1 & FN              & TN\\
		%			%& & P(y = 1) & P(y = 0)
		%		\end{tabular}
	%		
	%		\lz
	%		
	%		\centerline{Cost matrix }
	%		\begin{tabular}{cc|cc}
		%			& &\multicolumn{2}{c}{True class} \\
		%			& & $y=1$ & $y=-1$  \\
		%			\hline
		%			\multirow{2}{*}{\parbox{0.3cm}{Pred.  class}}& $\hat y$ = 1     & $C(1,1)$                & $C(1,-1)$\\
		%			& $\hat y$ = -1 & $C(-1,1)$              &  $C(-1,-1)$   \\
		%		\end{tabular}
	%		%\end{center}
	%		
	
	
	%	\end{itemize}



\begin{vbframe}{MetaCost: Overview}
%	
	\small{
	\begin{itemize}
%		
		\item Substantial work has gone into making individual algorithms cost-sensitive. A better solution would be to have a procedure that can convert a broad variety of cost-insensitive classifiers into cost-sensitive ones.
%		
		\item MetaCost is a wrapper method which can be used for \emph{any} type of classifier to obtain a cost-sensitive classifier. It treats the underlying classifier as a black-box in the sense that neither knowledge about its mechanism is required nor changes to its mechanism is needed.
%		
		\item MetaCost needs only a cost-matrix $\mathbf{C}$ for the underlying learning setting, which is used to adapt the decision boundaries towards predictions with low expected costs. \\
		{\scriptsize (Some tuning parameters are also needed.)}
%		
		\item Roughly speaking, the procedure of MetaCost is: relabel the training examples with their ``optimal'' classes, i.e., the ones with low expected costs, and apply the classifier on the relabeled data set.
		
%		
	\end{itemize}
%
	}
%	
\end{vbframe}


\begin{vbframe}{MetaCost: Algorithm}
	
	\scriptsize{
%		
 	The procedure of MetaCost is divided into three phases:
			%			
		\begin{minipage}{0.53\textwidth} 
%				
				\begin{enumerate}
%					
					\scriptsize
					\item Bagging --- The underlying classifier is used (trained) $L$ times on different bootstrapped samples of the training data, respectively.
%					
					\item Relabeling --- These $L$ trained classifiers are used to relabel the original training data set by taking the cost-matrix into account.
%					
					\item Cost-sensitivity ---  The classifier is trained on the relabeled data set resulting in a cost-sensitive classifier.
%					
				\end{enumerate}
%
	\scriptsize
	\lz 
			Predictions of the classifier $f$ are converted (if necessary) into probabilistic prediction:
			\begin{center}
				\tiny
							ProbPrediction$(j,f,\xv) = \begin{cases}
					(f(\xv))_j & \mbox{$f$ is a prob.\ classifier,} \\
					\mbox{One-hot$(f(\xv))$}& \mbox{else,}
				\end{cases}$
			\end{center}
%		
		\scriptsize
		where One-hot$(f(\xv))$ uses a one-hot-encoding of the prediction to make it a probability, i.e., one for the predicted class and 0 else. 
				%		
		\end{minipage}
		\begin{minipage}{0.45\textwidth} 
			\begin{algorithmic}
				
				\tiny
%				
				\State \textbf{MetaCost}  
				\State \textbf{Input:} 
				$\D = \{(\xi,\yi)\}_{i=1}^n$ training data, \\
				$L \in \N$ number of bagging iterations, \\
				$B \in \N$ bootstrap size, \\
				$f$ (black-box) classifier, 
				$\mathbf{C}$ cost matrix, 
%				$g = |\Yspace|$ number of classes
				\State \# 1st phase:
				\For{$l=1,\ldots,L$}
					\State $\D_l  \leftarrow $ BootstrapSample ($\D,B$)
					\State $f_l  \leftarrow $ train $f$ on $\D_l$
				\EndFor
				\State \# 2nd phase:
				\For{$i=1,\ldots,n$}
					\If{$\xi \in \D_l$ for all $l=1,\ldots,L$}
						\State $\tilde L \leftarrow \{1,\ldots,L\}$
					\Else
						\State $\tilde L \leftarrow \bigcup_{l: \xi \notin \D_l} \{l\}$
					\EndIf
					\For{$j=1,\ldots,g$} (relabel for binary case)
						\State $p_j(\xi)  \leftarrow \frac{1}{|\tilde L| } \sum_{l \in \tilde L}   p_j(\xi~|~ f_l) $
						\State $p_j(\xi~|~ f_l) = $ ProbPrediction$(j,f_l,\xi)$
						\State $\tilde y^{(i)} \leftarrow \argmin_{i^*} \sum_{j=1}^g p_j(\xi) C(i^*,j) $
					\EndFor
					\State $\tilde D \leftarrow \tilde D \cup \{(\xi,\tilde y^{(i)})\} $
				\EndFor
				\State \# 3rd phase:
%				
				\State $f_{meta} \leftarrow$ train $f$ on $\tilde D$
%			
			\end{algorithmic}
		\end{minipage}
		%
	}
	%	
\end{vbframe}


%\begin{vbframe}{MetaCost: Example}
%	%	
%	\small{
%		\begin{itemize}
%			%		
%			\item We compare C4.5 (decision tree) with MetaCost using C4.5 for the \href{http://staffwww.itn.liu.se/~aidvi/courses/06/dm/
%				labs/heart-c.arff}{heart data set}  in \href{ http://www.cs.waikato.ac.nz/ml/weka/}{Weka}. 
%			%		
%			\item The cost matrix $\mathbf{C}$ is 
%			
%			\begin{center}
%				\begin{tabular}{cc|cc}
%					& &\multicolumn{2}{c}{True class} \\
%					& & $y=1$ & $y=-1$  \\
%					\hline
%					\multirow{2}{*}{\parbox{0.3cm}{Pred.  class}}& $\hat y$ = 1     & $0$                & $ 1 $\\
%					& $\hat y$ = -1 & $ 4 $              &  $0$   \\
%				\end{tabular}
%			\end{center}
%		
%			\item The resulting confusion matrices are 
%			
%						
%			\begin{center}
%				\begin{tabular}{cc|cc}
%					& &\multicolumn{2}{c}{True class} \\
%					& MetaCost & $y=1$ & $y=-1$  \\
%					\hline
%					\multirow{2}{*}{\parbox{0.3cm}{Pred.  class}}& $\hat y$ = 1     & $104$                & $ 21 $\\
%					& $\hat y$ = -1 & $ 61 $              &  $117$   \\
%				\end{tabular}
%							\begin{tabular}{cc|cc}
%				& &\multicolumn{2}{c}{True class} \\
%				& C4.5 & $y=1$ & $y=-1$  \\
%				\hline
%				\multirow{2}{*}{\parbox{0.3cm}{Pred.  class}}& $\hat y$ = 1     & $138$                & $ 40 $\\
%				& $\hat y$ = -1 & $ 27$              &  $98$   \\
%			\end{tabular}
%			\end{center}
%		
%		
%			%		
%			\item The total cost of MetaCost is 145, while C4.5 has total costs of 187. However, MetaCost has $0.729$ correct classifications and C4.5 has $0.779.$ 
%			
%			%		
%		\end{itemize}
%		%
%	}
%	%	
%\end{vbframe}



%\begin{vbframe}{Cost-Sensitive Decision Trees}
%	%	
%	\small{
%		\begin{itemize}
%			%		
%			\item 
%			%		
%		\end{itemize}
%		%
%	}
%	%	
%\end{vbframe}


%
\endlecture
\end{document}
